\documentclass[12pt,a4paper]{article}
\usepackage[utf8]{inputenc}
\usepackage[francais]{babel}
\usepackage{amsmath}
\usepackage{amsfonts}
\usepackage{amssymb}
\usepackage{graphicx}
\usepackage{hyperref}
\graphicspath{ {image/} }
\usepackage[left=2cm,right=2cm,top=2cm,bottom=2cm]{geometry}

\bibliographystyle{ieeetran}

\usepackage{tikz}
\usetikzlibrary{shapes,arrows}

% styles for flowcharts
\tikzstyle{block} = [rectangle, draw, text width=10em, text centered, rounded corners, minimum height=4em,fill=blue!20]
\tikzstyle{cloud} = [draw, ellipse,fill=red!20, node distance=3cm,
    minimum height=2em]
%----------------------------------------------------------------------------------------
%	TITLE SECTION
%----------------------------------------------------------------------------------------

\newcommand{\horrule}[1]{\rule{\linewidth}{#1}} % Create horizontal rule command with 1 argument of height

\title{	
\normalfont \normalsize 
\textsc{Télécom Bretagne} \\ [25pt] % Your university, school and/or department name(s)
\horrule{0.5pt} \\[0.4cm] % Thin top horizontal rule
\huge Stage de fin d'étude \\ % The assignment title
\textsc{PLAN DU RAPPORT FINAL}
\horrule{2pt} \\[0.5cm] % Thick bottom horizontal rule
}

\author{HUYNH Le Duy} % Your name

\date{\normalsize\today} % Today's date or a custom date

\begin{document}

\maketitle % Print the title
\section{Introduction}
\subsection{Context of the internship}
\subsection{Objective and challenges}
\subsection{Proposed methodology}
\section{Theoretical Background}
\subsection{State-of-Art}
In this section, I will write about different approach in detecting text in natural images.
\subsection{Digital Topology and Self-Duality}
In this section, I will talk about digital topology and self-duality which links to the approach LRDE takes.
\subsection{Well-composed Images}
Then, I will follow by well-composed images, which has many interesting attribute for our approach.
\subsection{Tree-base presentation of image and connected filter}
In this section, I will talk about connected filter and tree-base presentation of image which have a special feature of reserving contour. This is the starting point for our approach. 
\section{Our approach}
\begin{figure}
 \begin{center}
  \begin{tikzpicture}[node distance=2 cm, auto, >=stealth]
   % nodes
   \node[cloud] (a) 				
   				{Entre};
 				
   \node[block,text width=5em] (b)  [right of=a, node distance=3cm]	
   				{gray-level images};
   \node[block, text width=8em] (c)  [right of=b, node distance=4cm]   
   				{Calculate gray morphological laplacian and gradient};

   \node[block, text width=8em] (d)  [right of=c, node distance=4.5cm]                       
   				{labeling composant};
   \node[block,text width=5em] (e)  [below of=d, node distance=4cm]                       
   				{Component grouping};
   \node[block] (f)  [left of=e, node distance=4cm]
                {Bounding box calculation};
   \node[block] (g)  [left of=f, node distance=5cm]
                {False positive elimination}; 
   \node[cloud] (h)  [left of=g, node distance=4cm]
                {output};                               
                   				

   % edges
   \draw[->] (a) -- (b);
   \draw[->] (b) -- (c);
   \draw[->] (c) -- (d);
   \draw[->] (d) -- (e);
   \draw[->] (e) -- (f);
   \draw[->] (f) -- (g);
   \draw[->] (g) -- (h);
  \end{tikzpicture}
  \caption{Chaine de détection de text a implémenter}
  \label{Process}
 \end{center}
\end{figure}
\section{Implementation}
\subsection{Well-composed Interpolation}
\subsection{Morphological laplacian and gradient}
\subsection{Labeling by front propagation}
\subsection{Average gradient of contour calculation}
\subsection{Component grouping and bounding box calculation}
\subsection{False positive elimination}
\section{Experimental results}
\subsection{Test image sets}
\subsection{Evaluation criteria}
\subsection{Results and comments}
\section{Optimization and results}
\section{Conclusions and perspectives}
\subsection{Conclusions}
\subsection{Perspectives}
\section{Bibliography}




	
\end{document}
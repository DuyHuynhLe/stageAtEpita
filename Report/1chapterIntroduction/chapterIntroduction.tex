
\graphicspath{ {1chapterIntroduction/image/} }
\chapter{Introduction}

With the increasing use of digital image capturing devices, such as digital cameras, mobile phones and PDAs, content-based image analysis techniques are receiving intensive attention in recent years. Among all the contents in images, text information has inspired great interests, since it can be easily understood by both human and computer, and has wide applications such as license plate reading, sign detection and translation, mobile text recognition, content-based web image search, and so on.


My internship took place during 6 months from mid Mars to mid September at LRDE, EPITA Research and Development Laboratory. I joint the Olena team. The subject of my internship is text detection in natural images using morphology tools. As the morphological tools has good properties in reserving shape, especially one class of morphological tool: connected operators, we arm to develop a text detection method that has smallest impact on text form. We also take advantage of the well-composed theory so that the algorithm could process dark-over-bright and bright-over-dark text the same way.


The first part of my internship is devoted to bibliography study. I learned the state-of-the-art method research state of art of text detection methods, digital topology and well-composedness and the less-famous morphology class of operator: connected operators. The second part of this internship was dedicated to design and implementation of the whole chain of text detection. In the final part, I optimized the performance of our proposed algorithm.

\graphicspath{ {7chapterConclusion/image/} }
\chapter{Conclusion}


In this work, an morphological approach, using connected operation and taking advantage of well-composedness, is presented and applied in the field of text detection. This approach was armed to be fast and reserved contour of text. This work consisted of two main stages. The first one is to design a text detection algorithm based on connected filter approach, using morphological Laplacian as feature to extract connected components. The second stage was spent with effort to improve the performance as well as the speed of this algorithm through optimization. 

In the first stage of this work, I have followed the connected filter approach, whose interesting properties is contour preserving to designed the text detection algorithm. The morphological Laplacian has been used as main feature to detect edge of component, together with gradient and other geometric attributes for elimination of false positive. This algorithm was designed to prune the tree of shape during its construction so this images only need to be read once, which reduce processing time. As a connected component approach, our work also provide the segmentation need by OCR. This approach shows promises as it gives a high recall rate, although the precision is low, these component can be later eliminated by OCRs in an end-to-end system.
  

The second stage of this work was contributed to improvement of precision and optimization of process speed. More criteria was added to eliminate false positives and to improve presicion rate 

This application of connected filter show promises as it obtains high recall rate and speed although more works must be done to improve the precision. 

 



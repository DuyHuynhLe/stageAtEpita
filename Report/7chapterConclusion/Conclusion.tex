
\graphicspath{ {7chapterConclusion/image/} }
\chapter{Conclusion and perspectives}

\section{Conclusion}
In this work, we propose a morphological approach based on connected operators and well-composedness for text detection. This approach was designed to be fast and preserve the contour of text. It is consisted of two main stages. The first one is to design a text detection algorithm based on connected filter approach. The second stage is to improve the performance and optimize the speed of this algorithm. 


In the first stage of this work, we used the connected filter approach, who has good contour preservation property to design the text detection algorithm. The morphological Laplacian has been used as main feature to detect edge of component, together with gradient and other geometric attributes to eliminate of false positives before word candidate grouping. This algorithm was designed to prune the tree of shape during its construction so the image is only loaded once to reduce processing time. As a connected component approach, our work also provides the segmentation required by OCR. 


This work provides complete text detection chain, character candidate isolation (through construction and pruning of the tree of shape of Laplacian) and word candidate grouping. The proposed approach is promising as it gives a high recall rate, although the precision is relative low, the false positives can be later eliminated by OCRs in an end-to-end system.
  

The second stage of our work was devoted to improve precision rate and optimize process speed. More criteria were added to eliminate false positives and to improve precision rate. For the false positives elimination parts, it has been shown that applying a Gaussian filter to eliminate noises slightly improves precision with small tradeoff the recall rate. Other criteria eliminating week contours and small objects also improve the performance. The optimization part was achieved by reducing memory usage and speed up image scanning part. The optimization makes our algorithm 36.84\% faster on average when applied to the ICDAR2013 focused scene text set. 

\section{Perspectives}

In further work, more study is needed to improve the precision rate because experiments show that simple criteria are not enough to remove false positive. We can apply a verification step by a learning machine or the OCR engine. 

Further work can also be done to improve the speed. In the current work, only the main loop was optimized and it improved the speed by 36.84\%. The contour following process still uses the old propagation form which is slow. It is possible to improve the execution time by further optimizing the labeling process by propagate using index of the data table for all process. Further more, the current method requires temporal storage of interpolated image, Laplacian and gradient which is used only in labeling process. We can re-implement this process using the interpolated information implicitly.


\graphicspath{ {0TitlePage/title/} }
\topskip0pt
\vspace*{\fill}

\renewcommand{\abstractname}{Acknowledgements}
\begin{abstract}
I would like to express my sincere thanks to Mr. Thierry Géraud at for giving me the opportunity to join his team as intern, for providing me all of the information, data and materials to support my work. Being my supervisor, he followed my work and gave me a lot of advice and useful remarks guiding me on the right way of this research work.

Besides my advisor, I would like to thank the rest of my Olena team: Edwin, Yongchao, Ana, Myriam and Jonathan, who always helps me and supports me. I thank my fellow labmates Alexandre, Krista, Divya et Nishit for all the fun we have after work.

At last, I would like to thank the Department of Image and Information Processing at Telecom Bretagne as well as EPITA Research and Development Laboratory for allowing me to do this interesting and enjoyable internship.

\end{abstract}
\vspace*{\fill}

\newpage

\topskip0pt
\vspace*{\fill}

\renewcommand{\abstractname}{Abstract}
\begin{abstract}

The aim of this internship is to propose a text detection processing chain that is based on mathematical morphology tools. Our method based connected filter that act by describing the input image by a hierarchical structure of disjoint sets and merging these set. These operations, which has very good contour preservation properties, ensure the conservation of character's shape. This approach uses morphological laplacian as main feature for the construction of the Tree of Shape representation of input image and morphological gradient and geometric criteria are used for pruning that tree. Nodes that share same parent are region having same background and will be formed text candidate if they have similar size and not very far apart. The propose approach is applied on the ICDAR 2013 Robust Reading Competition - Focused Scene Text database and evaluated by the DetEval protocol. It gains promising recall rate of 72.15\% and the average processing time is 3249.41 ms each image of the ICDAR2013 set. More improvement is needed to improve the execution time as well as the precision.
\end{abstract}

\selectlanguage{french} 
\begin{abstract}

Le but de ce stage est de proposer une chaîne de détection de texte qui est basé sur des outils de morphologie mathématique. Notre filtre connecté méthode basée qui agissent en décrivant l'image d'entrée par une structure hiérarchique des ensembles disjoints et la fusion de ces ensemble. Ces opérations, qui a de très bonnes propriétés de conservation de contour, peuvent assurer la conservation de la forme de caractère. Cette approche utilise Laplacien morphologique comme caractéristique principale pour la construction de l'arbre de la représentation de forme image d'entrée et le gradient morphologique et des critères géométriques sont utilisés pour l'élagage cet arbre. Nœuds qui partagent même parent sont région ayant même fond et sera formé texte candidat si elles ont une taille similaire et ne sont pas très éloignés. L'approche proposer est appliquée sur base de données de texte  du concours de lecture robuste d’ICDAR 2013 et évaluée par le protocole DetEval. Il gagne précision prometteur de 72.15\% et le temps moyen de traitement est 3249.41 ms par image de la base de ICDAR 2013. Plus l’amélioration est nécessaire pour réduire le temps d'exécution ainsi que la précision
\end{abstract}
\selectlanguage{english} 
\vspace*{\fill}
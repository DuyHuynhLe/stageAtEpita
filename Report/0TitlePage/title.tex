\graphicspath{ {0TitlePage/title/} }
\topskip0pt
\vspace*{\fill}

\renewcommand{\abstractname}{Acknowledgements}
\begin{abstract}
I would like to express my sincere thanks to Mr. Thierry Géraud at for giving me the opportunity to join his team as intern, for providing me all of the information, data and materials to support my work. Being my supervisor, he followed my work and gave me a lot of advice and useful remarks guiding me on the right way of this research work.

Besides my advisor, I would like to thank the rest of my Olena team: Yongchao, Edwin, Ana, Myriam and Jonathan, who always helps me and supports me. I thank my fellow lab mates Alexandre, Krista, Divya et Nishit for all the fun we have after work.

At last, I would like to thank the Department of Image and Information Processing at Telecom Bretagne as well as EPITA Research and Development Laboratory for allowing me to do this interesting and enjoyable internship.

\end{abstract}
\vspace*{\fill}

\newpage

\topskip0pt
\vspace*{\fill}

\renewcommand{\abstractname}{Abstract}
\begin{abstract}

The objective of this internship is to propose a text detection chain based on morphological tools. Our method based connected filter that act merging flat zone. It describes the input image by a hierarchical structure of disjoint sets and then merges these set. So it preserved well the character's shape thanks to the good contour preservation properties. We propose the construction of the Tree of Shape on the morphological Laplacian of the input image and use morphological gradient and geometric criteria to prune that tree. Nodes sharing the same parent are regions having the same background and will be formed as text candidate if they are close and have similar size. The proposed approach is validated on the ICDAR 2013 Robust Reading Competition - Focused Scene Text database and evaluated by the DetEval protocol. It achieved promising recall rate of 72.15\% and the average processing time is environs 3s for each image although the obtained precision is 6.11\%. The improvement of execution time and precision would be expected in the future work.
\end{abstract}

\selectlanguage{french} 
\begin{abstract}

Le but de ce stage est de proposer une chaîne de détection de texte basée sur des outils de morphologie mathématique. Nous avons utilisé les filtres connexes qui permettent de decrire l'image comme un ensemble de regions disjointes. En effet, les opérations de fusion de ces ensembles ont de très bonnes propriétés de conservation des contours, permettant d'assurer, ici la conservation de la forme des caractères de texte. Cette approche utilise principalement le Laplacien morphologique pour la construction de l'arbre des formes. Le gradient morphologique et des critères géométriques ont été utilisés pour l'élagage de cet arbre. Nous avons vu que des nœuds qui partagent les mêmes parents dans l'arbre, sont proches en terme de distance et sont similaires en terme de taille forment de potentielles chaînes de texte. Nous avons évalué notre approche sur la base de données (233 images) du concour ICDAR 2013 avec la métrique DetEval. Notre approche a obtenu un rappel de détection de 72.15\% et une précision de 6.11\%. Le temps moyen de traitement est d' environs 3s par image. Nos future travaux porterons sur l'élimination des faux positifs et la réduction du temp d'éxecution.
\end{abstract}
\selectlanguage{english} 
\vspace*{\fill}
\graphicspath{ {0TitlePage/title/} }
\section{Acknowledgements}
I would like to express my sincere thanks to Mr. Thierry Géraud at for giving me the opportunity to join his team as intern, for providing me all of the information, data and materials to support my work. Being my supervisor, he followed my work and gave me a lot of advice and useful remarks guiding me on the right way of this research work.

Besides my advisor, I would like to thank the rest of my Olena team: Edwin, Yongchao, Ana, Myriam and Jonathan, who always helps me and supports me. I thank my fellow labmates Alexandre, Krista, Divya et Nishit for all the fun we have after work.

At last, I would like to thank the Department of Image and Information Processing at Telecom Bretagne as well as EPITA Research and Development Laboratory for allowing me to do this interesting and enjoyable internship. 

\section{Abstract}
The aim of this internship is to propose an text detection processing chain that is based on mathematical morphology tools. Our method based connected filter that act by describing the input image by an hierarchical structure of disjoint sets and merging these set. These operations, which has very good contour preservation properties, ensure the conservation of character's shape. This approach uses morphological laplacian as main feature for the construction of the Tree of Shape representation of input image and morphological gradient and geometric criteria are use for pruning that tree. Nodes that share same parrent are region having same background and will be formed text candidate if they have similar size and not very far apart. The propose approach is applied on the ICDAR 2013 Robust Reading Competition - Forcused Scene Text data base and evaluated by the DetEval protocol. It gains promising precision of [] and the average processing time is[]


With the proliferation of mobile devices (phones, tablets, glasses), one can imagine new applications based on automatic reading text. A challenge is to extract the text in natural images or videos.
We propose a processing chain that is based on mathematical morphology tools. Schematically, this chain is broken down into three phases
\section{Résumé}

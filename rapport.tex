\documentclass[12pt,a4paper]{report}
\usepackage[utf8]{inputenc}
\usepackage[english]{babel}
\usepackage{amsmath}
\usepackage{amsfonts}
\usepackage{amssymb}
\usepackage{graphicx}
\graphicspath{ {image/} }
\usepackage[left=2cm,right=2cm,top=2cm,bottom=2cm]{geometry}
\bibliographystyle{ieeetran}

\author{HUYNH Le Duy}

\begin{document}

\tableofcontents

\chapter{Theoretical Background}
\section{Digital Topology and Self-Duality}
\paragraph{}
For a image defined in the regular cubical grid, the digital topology must be describe by a "Jordan pair" of connectivity $(c_\alpha,c_\beta)$ \cite{Kong:1989:DTI:71397.71400}.

\section{Well-composed Sets and Images}
\paragraph{}
As defined in \cite{Latecki95}, a 2D set S is weakly well-composed if any 8-component of S is a 4 component. S is well-composed if both S and its complement $\bar{C}$ are weakly well-composed. It can also be defined using the notion of "critical configurations" which are \includegraphics{confi1.jpg} and \includegraphics{confi2.jpg} : S is weakly well-composed if these configuration do not appear.
\paragraph{}
The notion of well-composednessed has also been extended to gray-level images. A gray-level image $\mu$ is well-composed if any set $[\mu \geq \lambda ]$ is well-composed. The extended version of "critical configuration" is that every block \begin{tabular}{|c|c|}
\hline 
a & d \\ 
\hline 
c & b \\ 
\hline 
\end{tabular} 
must verify interval(a,b) $\cap$ interval(c,d) $\neq \varnothing$ , where interval(a,b) = [min(a,b),max(a,b)].
\paragraph{}
An image is not a priori well-composed. There are 2 approach to get a well-composed image from the original image \cite{Geraud.15.ismm} by changing it pixel value (with possibility of alter the image's topology) or by a well-composed interpolation.

\section{Approach}
\subsection{The simple morphological tree-based framework for document representation}
\paragraph{}
The morphological tree can be use to represent the image contents. Using the information carried by that tree, we can keep only interesting elements such as text. To compute a representation tree, mathematical morphology requires at first an order function of pixel values. In the case of color images, since there is not a natural order of color, the definition of is important and affect the performance. A classical workaround is to sort using their luminance by converting color image to gray-level image. 
\paragraph{}
A first possible approach is to compute the tree of shapes of the luminance, since a visible text must have a “dark over light” contrast or the opposite one. Each note of this tree is a zone which have similar value.
\paragraph{}
A second possible approach base on the Laplacian morphology on the gray-level image. The zero-crossing of the laplacian are closed-contours and mark the boundary of object. By using laplacian morphology, we can distinguish object from the background. The size of objects conserved is depend on the windows using. Remind that the Laplacian morphology is calculate by $ \Delta_\Box = \delta_\Box + \varepsilon_\Box -2id $ as defined in \cite{Vliet_anedge}. The structure of the tree of shapes of the Laplacian is:
\begin{itemize}
\item A zone negative or positive will form a note
\item A zero point included in another note will belongs to that note
\item A note included in another note is a descendant of the former note
\end{itemize}
\paragraph{}
The tree of shapes of the Laplacian representation of region having similar luminance and their inclusion relationship. This tree is a light tree-base representation and well-suitable for some application such as image simplification [<give an example>] or smart binarization. 




\begin{thebibliography}{9}
\bibitem{Latecki95}
    {Longin Latecki and Ulrich Eckhardt and Azriel Rosenfeld},
    {Well-Composed Sets},
    {Computer Vision and Image Understanding},
    {1995},
    {61},
    {70--83}
    
\bibitem{Vliet_anedge}
	{Lucas J. Van Vliet and Ian T. Young and Guus L. Beckers},
    {AN EDGE DETECTION MODEL BASED ON NON-LINEAR LAPLACE FILTERING},
    Pattern Recognition and Artificial Intelligence,
    pp. 63-73,
    1988.  
\bibitem {Geraud.15.ismm}
  {Thierry G\'eraud and Edwin Carlinet and S\'ebastien Crozet},
  {Self-Duality and Digital Topology: Links Between the
		  Morphological Tree of Shapes and Well-Composed Gray-Level
		  Images},
  {Mathematical Morphology and Its Application to Signal and
		  Image Processing -- Proceedings of the 12th International
		  Symposium on Mathematical Morphology (ISMM)},
  {2015},
  {Lecture Notes in Computer Science Series},
  {9082},
  {Reykjavik, Iceland},
  {Springer},
  {J.A. Benediktsson and J. Chanussot and L. Najman and H.
		  Talbot},
  {573--584}.
\bibitem {Kong:1989:DTI:71397.71400}
	{Kong, T. Y. and Rosenfeld, A.},
	{Digital Topology: Introduction and Survey},
	{Comput. Vision Graph. Image Process.},
	{Dec. 1989},
	{48},
	{3},
	dec,
	{1989},
	{0734-189X},
	{357--393},
	{37},
	{Academic Press Professional, Inc.},
	{San Diego, CA, USA},    
    
\end{thebibliography}

\end{document}



